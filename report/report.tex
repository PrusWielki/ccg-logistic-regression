\documentclass[12pt]{article}

\usepackage[ margin=1in]{geometry}
\usepackage{titlesec}
\usepackage[hidelinks]{hyperref}
\usepackage{adjustbox}
\usepackage{float}
\usepackage{graphicx} %To include PDF as an image
\usepackage{times}
\usepackage{svg} % To include images in SVG format
\usepackage{ragged2e} % To justify abstract
\usepackage[T1]{fontenc}
\usepackage{comment} % For comment blocks
\usepackage[
    sorting=none
]{biblatex}
\usepackage{amsmath}
\usepackage{longtable}
\addbibresource{references.bib}


\titleformat*{\section}{\large\bfseries}

\begin{document}
\hypersetup{pdfauthor={Prusak, Patryk;}, pdftitle={}}

\title{Prusak_Patryk_ADVML_Project_1}
\author{\normalsize Patryk Prusak }
\hspace{0pt}
\vfill
\begin{center}

    \textbf{\huge{Cyclic Coordinate Descent for Logistic Regression with Lasso regularization}}
    
\end{center}

\vspace{0.5cm}
\begin{center}
    \normalsize{Patryk Prusak}
\end{center}
\vspace{0.5cm}
\begin{center}
    \normalsize{supervisor}\\
    \vspace{0.3cm}
    \normalsize{XYZ}
\end{center}

\vspace{0.5cm}

\begin{center}
\normalsize{Warsaw University of Technology} \\
\vspace{0.3cm}
\normalsize{\today} \\
\vspace{0.3cm}
Advanced Machine Learning Course
\end{center}

\vspace{1cm}


\tableofcontents
\vfill
\hspace{0pt}
\newpage

%TODO:
% Every point described below should be included in separate section. Maximal length of report is 6 pages A4 (title page and refernces are not included in the limit). Report should include:
% • Methodology.
% o Selection and generation of datasets.
% o Details about algorithm implementation and applied optimizations
% • Discussion about correctness of the LogRegCCD algorithm.
% Suggested approach to address this point:
% o Performance of the algorithm at lambda=0
% o Likelihood function values and coefficient values depending on iteration
% o Comparison with ready implementation of logistic regression with L1 penalty
% • Impact of dataset parameters: n.p,d,g on the performance of LogRegCCD algorithm.
% • Benchmark of LogRegCCD with LogisticRegression algorithm.
% Suggested approach to address this point:
% o Performance of algorithms regarding different metrics
% o Values of coefficients obtained in these two methods
% 1. 

\section{Methodology}

\subsection{Selection and generation of datasets}

\subsection{Details about algorithm implementation and applied optimizations}

\section{Discussion about correctness of the LogRegCCD algorithm}

\subsection{Performance of the algorithm at lambda=0}

\subsection{Likelihood function values and coefficient values depending on iteration}

\subsection{Comparison with ready implementation of logistic regression with L1 penalty}

\section{Impact of dataset parameters: n.p,d,g on the performance of LogRegCCD algorithm}

\section{Benchmark of LogRegCCD with LogisticRegression algorithm}

\subsection{Performance of algorithms regarding different metrics}

\subsection{Values of coefficients obtained in these two methods}


% \begin{figure}[h!]
%     \centering
%   \includegraphics[width=\textwidth]{images/bp4-tile-ferritin.png}
%     \caption{All downloaded ferritin proteins}
%     \label{fig:all-ferritin}
%   \end{figure}

\clearpage 
\listoffigures
\listoftables
\printbibliography

\end{document}
