\documentclass[12pt]{article}

\usepackage[ margin=1in]{geometry}
\usepackage{titlesec}
\usepackage[hidelinks]{hyperref}
\usepackage{adjustbox}
\usepackage{float}
\usepackage{graphicx} %To include PDF as an image
\usepackage{times}
\usepackage{svg} % To include images in SVG format
\usepackage{ragged2e} % To justify abstract
\usepackage[T1]{fontenc}
\usepackage{comment} % For comment blocks
\usepackage[
    sorting=none
]{biblatex}
\usepackage{amsmath}
\usepackage{longtable}
\addbibresource{references.bib}


\titleformat*{\section}{\large\bfseries}

\begin{document}
\hypersetup{pdfauthor={Prusak, Patryk;}, pdftitle={}}

\title{Prusak_Patryk_ADVML_Project_1}
\author{\normalsize Patryk Prusak }
\hspace{0pt}
\vfill
\begin{center}

    \textbf{\huge{Cyclic Coordinate Descent for Logistic Regression with Lasso regularization}}
    
\end{center}

\vspace{0.5cm}
\begin{center}
    \normalsize{Patryk Prusak}
\end{center}
\vspace{0.5cm}
\begin{center}
    \normalsize{supervisor}\\
    \vspace{0.3cm}
    \normalsize{XYZ}
\end{center}

\vspace{0.5cm}

\begin{center}
\normalsize{Warsaw University of Technology} \\
\vspace{0.3cm}
\normalsize{\today} \\
\vspace{0.3cm}
Advanced Machine Learning Course
\end{center}

\vspace{1cm}


\tableofcontents
\vfill
\hspace{0pt}
\newpage

\section{Objective}
The report describes conducting analyses using the UCSF Chimera program on proteins, including visualization, alignment, and similarity comparison.


% \begin{figure}[h!]
%     \centering
%   \includegraphics[width=\textwidth]{images/bp4-tile-ferritin.png}
%     \caption{All downloaded ferritin proteins}
%     \label{fig:all-ferritin}
%   \end{figure}

\clearpage 
\listoffigures
\listoftables
\printbibliography

\end{document}
